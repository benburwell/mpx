\chapter{Overview of COMHAN}
\label{overview_of_comhan}

\section{The Command Handler}

All of your interactions with MPX-OS \index{MPX-OS} will be through the
``Command Handler,'' \index{Command Handler} referred to
as ``comhan.'' The command handler is a process that is loaded when MPX-OS first starts
and allows users to execute commands such as getting help, running programs, setting the
date, and other system-wide functionality.

To use the command handler, you must only start MPX. You will see the prompt {\tt mpx>}
which indicates that the system is ready to accept commands. You may then type in a 
command on the keyboard and press return. The command handler will then attempt to parse 
your command. If what you have entered is a valid system command, MPX will execute your 
request and then prompt you for the next command. If the text you enter at the MPX prompt
is not valid, you will see an error message.

Note that the command handler parses your input case insensitively; that is, typing 
{\tt version}, {\tt VERSION}, and {\tt vERsiOn} at the prompt will result in identical 
output.

All of the commands used in MPX-OS are listed in \ref{summary_of_commands}. Alternately,
when in MPX-OS, the command {\tt help} will display a list of commands and a brief usage
description. Of particular note, however, is the {\tt stop} command, which will terminate
MPX-OS and return control of the computer to DOS.

\section{A Note on Errors}

When writing MPX-OS, we have subscribed to the philosophy that the user does not need to 
know what the system is doing unless it pertains to their actions. Thus, for the most part,
the normal behavior of a command is to see no output. For example, when changing the 
prompt (see p. \pageref{prompt_cmd}), there is no ``prompt changed'' message; the user
should be able to trust the system to perform correctly.

Thus, messages will generally only be displayed when some unexpected behavior has occurred. 
For full and more precise details on exactly what the output of each command, see the
appropriate documentation in this Manual.
