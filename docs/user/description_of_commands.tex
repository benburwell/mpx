\chapter{Detailed Description of Each Command}
\label{description_of_commands}

\section{\tt version}
\index{version@{\tt version}}
\label{version_cmd}

This command will display some information about the current version of MPX that is
running. 

\subsection{Syntax}
The command has the form {\tt version}.

\subsection{Usage}
The command can be called from the MPX prompt at any time.

\subsection{Example}
\begin{enumerate}
    \item To get the version of MPX-OS that is currently running: \\
        {\bf Input:} {\tt version} \\
        {\bf Output:} {\tt MPX OS - Version 1.0}
\end{enumerate}

\subsection{Possible Errors}
There are no error conditions.








\section{\tt date}
\index{date@{\tt date}}
\label{date_cmd}

This command is used to retrieve and set the system date. Note that the date does not
actually increment as time passes.

\subsection{Syntax}
The command optionally takes a date string formatted as {\tt mm/dd/yyyy} where {\tt mm}
is a two-digit one-based integer representing the month ({\tt 01} corresponding to January,
{\tt 02} to February, and so on), {\tt dd} similarly corresponding to a one-based date of
month, and {\tt yyyy} a year.

The command has the form {\tt date [mm/dd/yyyy]}.\footnote{Note that in this document, we 
use {\tt [ ]} to denote optional parameters.}

\subsection{Usage}
The command can be called from the MPX prompt at any time.

\subsection{Example}
\begin{enumerate}
    \item To get the current system date: \\
        {\bf Input:} {\tt date} \\
        {\bf Output:} {\tt 12/04/2013}
    \item To set the system date: \\
        {\bf Input:} {\tt date 12/04/2013} \\
        {\bf Output:} None.
    \item Attempting to set an invalid date: \\
        {\bf Input:} {\tt date 99/99/9999} \\
        {\bf Output:} {\tt Invalid date.}
\end{enumerate}
\subsection{Possible Errors}
The message ``Invalid date'' will be displayed if the date is not in a valid format, i.e.
the month is not between 1 and 12 or date not between 1 and 31, year not a number.








\section{\tt directory}
\index{directory@{\tt directory}}
\label{directory_cmd}

The directory command will display a listing of all available MPX programs that can be
loaded and run, along with their sizes.

\subsection{Syntax}

The command has the form {\tt directory}.

\subsection{Usage}

The command can be called from the MPX prompt at any time.

\subsection{Example}
\begin{enumerate}
    \item Get a listing of programs when there are none: \\
        {\bf Input:} {\tt directory} \\
        {\bf Output:} {tt You have no programs}
    \item Get a listing of programs when there is one: \\
        {\bf Input:} {\tt directory} \\
        {\bf Output:} {tt You have 1 program}
    \item Get a listing of programs when there are four: \\
        {\bf Input:} {\tt directory} \\
        {\bf Output:} {tt You have 4 programs}
\end{enumerate}

\subsection{Possible Errors}
This command will not generate any errors.








\section{\tt stop}
\index{stop@{\tt stop}}
\label{stop_cmd}

The stop command allows the user to exit MPX-OS and return to DOS. When the command is
issued, the user will be prompted to confirm their action by typing the character {\tt y}.
If any other character is entered, the user will remain in MPX-OS.

\subsection{Syntax}

The command has the form {\tt stop}.

\subsection{Usage}

The command can be called at any time through the MPX prompt.

\subsection{Example}
\begin{enumerate}
    \item User calls stop, but does not want to exit: \\
        {\bf Input:} {\tt stop} \\
        {\bf Output:} {\tt Are you sure you want to exit? [y/n]:} \\
        {\bf Input:} {\tt n} \\
        {\bf Output:} None.
    \item User calls stop, and does confirm wanting to exit: \\
        {\bf Input:} {\tt stop} \\
        {\bf Output:} {\tt Are you sure you want to exit? [y/n]:} \\
        {\bf Input:} {\tt y} \\
        {\bf Output:} {\tt ** COMHAN execution complete **}

\end{enumerate}
\subsection{Possible Errors}

There are no error conditions.








\section{\tt help}
\index{help@{\tt help}}
\label{help_cmd}

The help command allows the user to get help about using MPX commands. If the command is
run with no options, a brief description of all commands. Alternately, if called with the
name of a command as an option, it will give only the description of that command.

\subsection{Syntax}

The command has the form {\tt help [command]}.

\subsection{Usage}

The command can be called at any time through the MPX prompt.

\subsection{Example}
\begin{enumerate}
    \item User wants help for all processes: \\
        {\bf Input:} {\tt help} \\
        {\bf Output:} \\
        \begin{tabular}{ll}
            {\tt Name                 } & \tt{Use                                              } \\
            {\tt ==================== } & \tt{=================================================} \\
            {\tt version              } & \tt{Display version number                           } \\
            {\tt help                 } & \tt{Provide information about commands               } \\
            {\tt directory            } & \tt{List .mpx files                                  } \\
            {\tt date [mm/dd/yyyy]    } & \tt{Display or set the system date                   } \\
            {\tt stop                 } & \tt{Terminate execution of COMHAN                    } \\
            {\tt prompt string        } & \tt{Change the prompt for commands                   } \\
            {\tt alias command=string } & \tt{Create an alias for a command                    } \\
            {\tt show                 } & \tt{Prints PCB information                           } \\
            {\tt allocate             } & \tt{Builds PCB with specified options                } \\
            {\tt free name            } & \tt{Frees the PCB called name                        } \\
            {\tt load name[=ppp]      } & \tt{Creates a process called name with priority ppp  } \\
            {\tt resume name          } & \tt{Resumes the process called name                  } \\
            {\tt run name[=ppp]       } & \tt{Runs a process called name with priority ppp     } \\
            {\tt suspend name         } & \tt{Suspends the process called name                 } \\
            {\tt terminate name       } & \tt{Terminates the process called name               } \\
            {\tt setpriority name=ppp } & \tt{Sets the priority of process name                } \\
            {\tt dispatch             } & \tt{Runs each process once                           } \\
            {\tt clock [stop|start]   } & \tt{Perform clock operations                         } \\
        \end{tabular}

    \item User wants help for the {\tt show} command: \\
        {\bf Input:} {\tt help show} \\
        {\bf Output:} \\
        \begin{tabular}{ll}
            {\tt Name                 } & \tt{Use                                              } \\
            {\tt ==================== } & \tt{=================================================} \\
            {\tt show                 } & \tt{Prints PCB information                           } \\
        \end{tabular}

\end{enumerate}
\subsection{Possible Errors}

There are no error conditions for this command.








\section{\tt prompt}
\index{prompt@{\tt prompt}}
\label{prompt_cmd}

This command is used to alter the prompt that is displayed to the user by the command 
handler. For example, the user could change the prompt from the default {\tt mpx>} to
{\tt >>}.

\subsection{Syntax}

The command has the form {\tt prompt string}, where {\tt string} is the new prompt.

\subsection{Usage}

The command can be called from the MPX prompt at any time.

\subsection{Example}
\begin{enumerate}
    \item User wants to set the prompt to ``os'': \\
        {\bf Input:} {\tt prompt os} \\
        {\bf Output:} No output.
\end{enumerate}

\subsection{Possible Errors}

If a {\tt string} is not passed, the user will receive the error ``No prompt specified.''








\section{\tt alias}
\index{alias@{\tt alias}}
\label{alias_cmd}

The alias command allows the user to make an alias for any of the commands. For example, 
if the user needs to access the value of the system clock in rapid succession, the
{\tt clock} command could be aliased to simply {\tt c}.

\subsection{Syntax}

The syntax for this command is: {\tt alias command=new}, with {\tt command} the name of
the command to be aliased, and {\tt new} the string that should be made equivalent to
{\tt command}.

\subsection{Usage}

The command can be called from the MPX prompt at any time.

\subsection{Example}
\begin{enumerate}
    \item Setting ``c'' to be an alias for ``clock'': \\
        {\bf Input:} {\tt alias clock=c} \\
        {\bf Output:} None.
\end{enumerate}
\subsection{Possible Errors}

If {\tt command} or {\tt new} are not specified, the user will receive the error ``No
prompt specified.''








\section{\tt show}
\index{show@{\tt show}}
\label{show_cmd}

The show command is used to view the status of various processes.

\subsection{Syntax}

The command is of the form {\tt show type}, where type is one of {\tt free}, {\tt all},
{\tt system}, {\tt application}, {\tt suspended}, {\tt ready}, or {\tt init}.

\subsection{Usage}

\begin{enumerate}
    \item[{\tt free}] Display all free process control blocks.
    \item[{\tt all}] Display all process control blocks.
    \item[{\tt system}] Display all system processes.
    \item[{\tt application}] Display all application processes.
    \item[{\tt suspended}] Display all suspended processes.
    \item[{\tt ready}] Display all ready processes.
    \item[{\tt init}] Display all processes that are waiting for I/O.
\end{enumerate}

If a type is not given by the user, a message describing the possible values is displayed.

\subsection{Example}
\begin{enumerate}
    \item User wants a listing of all process control blocks: \\
        {\bf Input:} {\tt show all} \\
        {\bf Output:} A table of data about all process control blocks.
\end{enumerate}
\subsection{Possible Errors}

If the type specified is not one of the valid types, an empty table will be displayed.







\section{\tt load}
\index{load@{\tt load}}
\label{load_cmd}

This command loads a program file into memory and prepares it for execution by MPX-OS.

\subsection{Syntax}

The command has the form {\tt load proc[=ppp]} where {\tt proc} is the name of the process
to be loaded (and stored in the file {\tt proc.mpx}), and {\tt ppp} is an optional
execution priority (default is zero). The priority $p$ must satisfy $-127 < p < 127$.

\subsection{Usage}

The command can be called from the MPX prompt at any time.

\subsection{Example}
\begin{enumerate}
    \item User wants to load the process ``proc1'': \\
        {\bf Input:} {\tt load proc1} \\
        {\bf Output:} None.
    \item User wants to load the process ``proc2'' with priority 100: \\
        {\bf Input:} {\tt load proc2=100} \\
        {\bf Output:} None.
\end{enumerate}
\subsection{Possible Errors}

Several errors and warnings can occur when calling the command. If there is no file named
{\tt proc.mpx} where {\tt proc} is the name argument to the load command, the message 
``Error: No program found with that name'' will be displayed and the program will not be 
loaded.

If the program file does exist, but MPX-OS is unable to allocate sufficient memory for it
to be loaded, the message ``Error: Insufficient memory'' will be displayed and the program
will not be loaded.

If there is sufficient memory, but the system encounters an error while loading the
program file into memory, the message ``Error: Could not load program into memory'' will 
be displayed, and again, the program will not be loaded.

If an invalid priority was specified by the user, the message ``Warning: invalid priority
specified, using default'' will be displayed. MPX-OS will continue to load the program,
but with zero as the priority.

If there are no available process control blocks to allocate to the process, MPX cannot
run the program at this time. In this case, the message ``Error: No free PCBs'' will be
displayed and the program will not be loaded.

In the case that there is an error building the process control block for the process that
has been loaded into memory, MPX-OS will not be able to execute it. Thus, the memory will
be deallocated and the message ``Error: Unable to build PCB'' will be displayed to the
user.






\section{\tt resume}
\index{resume@{\tt resume}}
\label{resume_cmd}

The resume command allows a suspended process to continue running.

\subsection{Syntax}

The resume command has the format {\tt resume name|*}.\footnote{Note that we are using {\tt |} to denote ``or.''} If a {\tt name} is given, MPX-OS 
will attempt to find a process named {\tt name} and resume it. Otherwise, if {\tt *} is 
given, MPX-OS will resume all currently suspended processes.

\subsection{Usage}

The resume command is used when the user has suspended a process and wishes it to resume
its execution.

\subsection{Example}
\begin{enumerate}
    \item User wants to resume the process {\tt proc}: \\
        {\bf Input:} {\tt resume proc} \\
        {\bf Output:} None.
    \item User wants to resume all processes: \\
        {\bf Input:} {\tt resume *} \\
        {\bf Output:} None.
\end{enumerate}
\subsection{Possible Errors}

When {\tt resume} is called with no options, a message reminding the user to enter the
name of a process to resume or an asterisk is displayed.

If a name was given as the option, but MPX-OS cannot find a process with that name, the
message ``No process named $x$'' will be displayed.

Upon correct execution, there will be no output.





\section{\tt run}
\index{run@{\tt run}}
\label{run_cmd}

The run command behaves similarly to the load command, with the difference being quite
minor. Whereas the load command simply allocated memory and set up a process control block
for a process to be later run by a user using the resume command, the run command
compresses these steps into one operation. Put simply, the run command does just that: it
load a program from the disk and runs it.

\subsection{Syntax}

The command has the form {\tt run proc[=ppp]} where {\tt proc} is the name of the program
(i.e., there is a file named {\tt proc.mpx}) and {\tt ppp} is the priority that should be
used, where {\tt ppp} satisfies the condition $-127 < {\tt ppp} < 127$.

\subsection{Usage}
Usage
\subsection{Example}
\begin{enumerate}
    \item User wants to run the process ``proc1'': \\
        {\bf Input:} {\tt run proc1} \\
        {\bf Output:} None.
    \item User wants to run the process ``proc2'' with priority 100: \\
        {\bf Input:} {\tt run proc2=100} \\
        {\bf Output:} None.
\end{enumerate}
\subsection{Possible Errors}

Since the run command first loads the process, all of the error messages from the load
command can be obtained by using the run command.

In addition, if there was an error loading the process, the message ``Error: process did
not load correctly'' will be displayed.







\section{\tt suspend}
\index{suspend@{\tt suspend}}
\label{suspend_cmd}

The suspend command is used when a running program needs to be temporarily paused.

\subsection{Syntax}

The command has the form {\tt suspend proc|*} where {\tt proc} is the name of the program
to be suspended. If the {\tt *} option is used, all running application processes will be 
suspended.

\subsection{Usage}
The command can be called whenever there is a running program.

\subsection{Example}
\begin{enumerate}
    \item User wants to suspend the process {\tt proc}: \\
        {\bf Input:} {\tt suspend proc} \\
        {\bf Output:} None.
    \item User wants to suspend all processes: \\
        {\bf Input:} {\tt suspend *} \\
        {\bf Output:} None.
\end{enumerate}
\subsection{Possible Errors}

If there is no process named {\tt proc}, the message ``No process with the specified name''
will be displayed.







\section{\tt terminate}
\index{terminate@{\tt terminate}}
\label{terminate_cmd}

The terminate command frees the PCB that was associated with an application process and
frees the memory that was in use by the program.

\subsection{Syntax}

Similarly to the resume and suspend commands, the format is {\tt terminate name|*}, where
{\tt name} corresponds to the name of the application process to be terminated, and {\tt *}
to the fact that all application processes should be terminated.

\subsection{Usage}

The command can be run when there are currently running or suspended application processes.

\subsection{Example}
\begin{enumerate}
    \item User wants to terminate the process {\tt proc}: \\
        {\bf Input:} {\tt terminate proc} \\
        {\bf Output:} None.
    \item User wants to terminate all processes: \\
        {\bf Input:} {\tt terminate *} \\
        {\bf Output:} None.
\end{enumerate}
\subsection{Possible Errors}

If there is no process called {\tt name}, the message ``No process with the specified name''
will be displayed.







\section{\tt setpriority}
\index{setpriority@{\tt setpriority}}
\label{setpriority_cmd}

The setpriority command allows the user to specify an execution priority for a process.

\subsection{Syntax}
The format of the command is {\tt setpriority name=ppp} where {\tt name} refers to the 
name of the affected process and {\tt ppp} to the priority. As only the priority of 
application processes can be set, $-127 < {\tt ppp} < 127$ must be satisfied.

\subsection{Usage}
The command can be called when there is a non-zero number of application processes in MPX.

\subsection{Example}
\begin{enumerate}
    \item Set the priority of programm {\tt proc} to be $-90$: \\
        {\bf Input:} {\tt setpriority proc=-90} \\
        {\bf Output:} None.
\end{enumerate}
\subsection{Possible Errors}

There are two possible errors from this command. First, the priority could be invalid, in
which case the message ``Error: invalid priority'' will be displayed. In the other, there
is not really any process with the name specified, and MPX-OS will display ``Error: invalid
process name.''







\section{\tt clock}
\index{clock@{\tt clock}}
\label{clock_cmd}

The clock command is used to perform various tasks using the system clock. The clock can
be stopped and started, set to a particular value, or have its value read.

\subsection{Syntax}

The command uses the form {\tt clock job|time}, where {\tt job} is one of {\tt stop}, 
{\tt start} and {\tt time} is a string formatted as {\tt hh:mm:ss} with {\tt hh} hours, 
{\tt mm} minutes, and {\tt ss} seconds.

\subsection{Usage}

The clock command can be called at any time from the MPX-OS command handler.

\subsection{Example}
\begin{enumerate}
    \item Starting the clock: \\
        {\bf Input:} {\tt clock start} \\
        {\bf Output:} None.
    \item Stopping the clock: \\
        {\bf Input:} {\tt clock stop} \\
        {\bf Output:} None.
    \item Getting the clock value: \\
        {\bf Input:} {\tt clock} \\
        {\bf Output:} {\tt The current time is 21:09:44}
    \item Setting the system clock: \\
        {\bf Input:} {\tt clock 21:09:44} \\
        {\bf Output:} None.
\end{enumerate}
\subsection{Possible Errors}

If for some reason, the clock is unable to be set, the message ``Error setting clock'' 
will be displayed.








